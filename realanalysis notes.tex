\documentclass[12pt,letterpaper]{article}


\usepackage{amsmath}
\usepackage{amsfonts}
\usepackage{amsthm}
\usepackage{ulem}


\title{Real Analaysis Notes}
\author{Håkon Berggren Olsen\\
  \small{MAT2400- Real Analisys}\\
  \small{University of Oslo}\\
  \small{hakonberggren@gmail.com}
}
\date{Spring 2022}

\begin{document}
 
\maketitle



%%%%%%%%%%%%%%%%%%%%%%%%%%%%%%%%%%%%%%%%%%%%%%%%%%%%%%
\section*{De Morgan's Laws}
These laws state that the complement of 
\begin{align*}
	\text{(i)} \quad  (A_1 \cup A_2 \cup... \cup A_n)^c = A_1^c \cap A_2^c \cap ... \cap...A_n^c \\
	\text{(ii)} \quad  (A_1 \cap A_2 \cap... \cap A_n)^c = A_1^c \cup A_2^c \cup ... \cup...A_n^c \\
\end{align*}

%%%%%%%%%%%%%%%%%%%%%%%%%%%%%%%%%%%%%%%%%%%%%%%%%%%%%%
\section*{Families of sets}
A collection of sets, think set of sets, is usually called a family. An example is the family

\begin{align*}
	\mathbb{A} = \{[a,b] | a,b \in \mathbb{R}\}
\end{align*}
of all closed and bounded intervals on the real line. \\


\noindent \\
\section*{Functions}
If $\mathbf{A}$ is a subset of $X$, the set $f(A) \subset Y$ defined by


\begin{align*}
	f(A) = \{ f(a) | a \in A\}
\end{align*}

is called the \textit{images of $A$ under $f$}. 

\noindent \\
If $B$ is a subset of $Y$, the set $f^{-1}(B) \subset X$ is defined by

\begin{align*}
	f^{-1}(B) = \{ x | x \in B\}
\end{align*}
Note that the inverse function only is defined when the funciton is bijective. However the inverse images $f^{-1}(B)$ that is studied above are defined for all functions $f$.

\section*{Relations and partitions}
Relations are an abstract way of relating something to each other. Tangible examples of this can be the difference in magnitude (denoted by less than or greater than signs), angle(s) between vectors, similiar matricies and properties thereof. Below is an abstract defintions of such relations. \\

\noindent
$\bold{Defintion}$: By a relation on a set $X$, we mean a subset $R$ of the carteisan product $X \times X$. We usually write $xRy$ instead of $(x,y) \in \mathbb{R}$ to denote that $x$ and $y$ are related. The symbols $\sim$ and $\equiv$  \footnote{the \LaTeX\  symbol for these signs are ''sim'' and ''equiv''} are often used to denote relations, and we then write $x\sim y$ and $x \equiv y$.

$\bold{Example}$: Equality (denoted by the symbol $=$) and less than ($<$)  are relations on $\mathbb{R}$. To see that they fit into the formal definition above, note that they can be defines as 
\begin{align*}
	R &= \{(x,y) \in \mathbb{R}^2 | x = y\} \\
	S &= \{(x,y) \in \mathbb{R}^2 | x < y\} 
\end{align*}

Partion is a division of sets into nonoverlapping pieces. More preciesly, if $X$ is a set, a partition $\mathbb{P}$ of $X$ is a family of nonempty subsets of $X$ such that each element in $x$ belongs to exactly one set $P \in \mathbb{P}$. These sets $P \in \mathbb{P}$ are called parition classes of $\mathbb{P}$. 

$\bold{Example}$:  Given a parition of $X$, we may introduce a relation $\sim$ on $X$ by

\begin{align*}
	x \sim y \Leftrightarrow x \text{ and } y \text{ belong to the same set $P \in \mathbb{P}$.}.
\end{align*}


\noindent  
Equivalence relations - Used to partion sets into subsets.

$\bold{Defintion}$: An equvialnce relation on $X$ is a relation $\sim$ satisfying the follonw conditions:
\begin{itemize}
	\item Reflexivity: $x\sim x$ for all $x\in X$.
	\item Symmetry: If $x\sim y$, then $y\sim x$.
	\item Transistivity: If $x\sim y$ and $y\sim z$ then $x\sim z$.
\end{itemize}

\section*{Countability}
A set $A$ is called countable if it is possible to make a list $a_1, a_2, ..., a_n,...$ which contains all elements of $A$. If this isn't possible, the set is called countable. 
The infinte countable sets are the smalles infintes sets, and such this is a way to give a ''magnitude'' to infinity. Note that $\mathbb{R}$ is too large to be uncountable.

\noindent
Finites sets are obivously countable and so are the subsets of finite sets.

\noindent
The set of natural numbers $\mathbb{N}$ is also countable, listed as $1,2,3,...$, and such is the set of integers $\mathbb{Z}$ aswell, however this is less obvious.

\noindent

$\bold{Propostiion}$ If the sets $A,B$ are countable, so is the cartesian product $A \times B$.



%%%%%%%%%%%%%%%%%%%%%%%%%%%%%%%%%%%%%%%%%%%%%%%%%%%%%%
\section*{Completeness}


\textbf{The Completeness Principle} \\
\noindent
Assume that $A$ is a nonempty subset of $\mathbb{R}$ (i.e. $A \subset \mathbb{R}, \ $A$ \neq \emptyset$). We say that $A$ is \emph{bounded above} if there is a number $b\in\mathbb{R}$ such that $b\ge a$ for all $a\in A$, and we say that $A$ is \emph{bounded below} if there is a number $c\in\mathbb{R}$ such that $c\le a$ for all $a\in A$. We call $b$ and $c$ an\emph{upper} and a \emph{lower bound} of $A$, respectively.\\
\noindent

\textbf{THe Completeness Principle:} Every nonempty subset $A$ of $\mathbb{R}$ that is bounded above has a least upper bound. \\

Least upper bound of $A$ is also called the \emph{supremum} of $A$ and is denoted by 
\begin{align*}
	\text{sup} A
\end{align*}



Some key defintions for central terms are 
\begin{itemize}
	\item \textbf{Bounded} - This just means there is a limit how big (or how small, if one is looking at bounded below) a number in a sequence can get. 
	\item \textbf{Supremum} - This is the smallest possible upper bound.There are many bounds one can set, but this is the closest one can get to the upper bound. 
	\item \textbf{Infinitum(?)} - This is the largest possible lower bound. There are many bounds one can set, but this is the closest one can get to the lower bound
	\item $\bold{Bounded-above}$- A is bounded above if there is a number $b\in\mathbf{R}$ such that $b\ge a$ for all $a\in A$.
	\item $\bold{Bounded-above}$ - 
	\item $\bold{Bounded-below}$ - A is bounded below if there is a number $c\in\mathbf{R}$ such that $c\le a$ for all $a\in A$.
	\item $\bold{Bounded-above}$ - 
\end{itemize}








\section*{Cauchy Sequences}
\uline{\textbf{Definition:}} A sequence $\{x_n\}$ of poitns in $\mathbb{R}^d$ is called a \uline{Cauche-sequence} if for every $\epsilon>0$, there is an $N\in\mathbb{N}$ (or viewed as $N(\epsilon)$, such that for all $n,m\ge M$we have  $||\vec{x}_n-\vec{x}_m||<\epsilon$.\\
\\
\uline{\textbf{Lemma:}}All convergent sequences are Cauchy sequencess. \\
\\
\uline{\textbf{Proof:}} Since $\{\vec{x}_n\}$ converges to a point $\vec{x}$, these are for any $\epsilon >0$ and $N\in\mathbb{N}$ such that if $n\ge N$, then $||\vec{x}_n-\vec{m}||<\frac{\epsilon}{2}$. Hence if  $n,m \ge N$, then 

\begin{align*}
	||\vec{x}_n-\vec{x}_m|| = ||(\vec{x}_n- \vec{x}_m) + ( \vec{x}- \vec{x}_m)|| \le ||(\vec{x}_n- \vec{x})|| + ||( \vec{x}- \vec{x}_m)|| \le \frac{\epsilon}{2}+\frac{\epsilon}{2} = \epsilon \\
\end{align*}

%%%%%%%%%%%%%%%%%%%%%%%%%%%%%%%%%%%%%%%%%%%%%%%%%%%%%%
\section*{Intermediate Value Theorem}






%%%%%%%%%%%%%%%%%%%%%%%%%%%%%%%%%%%%%%%%%%%%%%%%%%%%%%
\section*{The Bolzano-Weierstrass Theorem}
\uline{A Preview of Compactness}

Given a sequence $\{\vec{x_m}\}$ is in $\mathbb{R}^d$.
\begin{align*}
	\vec{x}_1, \vec{x}_2, \vec{x}_3, \vec{x}_4, \vec{x}_5, \vec{x}_6, \vec{x}_7, \vec{x}_8, \vec{x}_9, ... \\
\end{align*}
\uline{Subsequence:} $ \vec{x}_2, \vec{x}_5, \vec{x}_7, \vec{x}_9, ...$ \\

\uline{Formally}: Let $n_1<n_2<n_3<....$ be a strictly increasing sequences of natural numbers. The sequence

\begin{align*}
	\vec{x}_{n_1}, \vec{x}_{n_2}, \vec{x}_{n_3}...
\end{align*}
is called a subsequence of $\{\vec{x}_n \}$.



\uline{Bolzano-Weierstrass Theorem}: All bounded sequences in $\mathbb{R}^d$ have convergent subsequences. 
Proof for $d=2$ (2-dimensional case). 




%%%%%%%%%%%%%%%%%%%%%%%%%%%%%%%%%%%%%%%%%%%%%%%%%%%%%%
\section*{The Extreme Value Theorem}
If $f:[a,b]\to\mathbb{R}$ is continous defined on a closed and bounded interval, then $f$ has maximum and minimum points on $[a,b]$.

graph here \\
\\
\uline{Proof:} Let
\begin{align*}
	A = \{ f(x) : x\in[a,b]\}
\end{align*}
and put $M = \text{sup} A$ ( with $M=\infty$ if $A$ is upward bounded). \\

Pick a sequence $\{x_n\}$ from $[a,b)$ such that $f(x_n)\to M$.\\

By Bolzano-Weierstrass Theorem, $\{x_n\}$ has a convergent subsuequence $\{x_{n_k}\}$, i.e. $C = \text{lim } x_{u_k}$.\\

Since $[a,b]$ is closed $c\in [a,b]$. Since $f$ is continous $x_{u_k}\to C$ implies $f(x_{u_k})\to f(c)$.\\

 But we also $f(x_{u_k}) \to M$.\\

\begin{align*}
	f(x_n) \to M
\end{align*}













%%%%%%%%%%%%%%%%%%%%%%%%%%%%%%%%%%%%%%%%%%%%%%%%%%%%%%
\section*{The Mean Value Theorem}
Assume that $f : [a,b] \to \mathbf{R}$ is continous in all of $[a,b]$ and differentiable at all inner points $x \in (a,b)$. Then there is a point $c \in (a,b)$ such that 

\begin{align*}
	f'(c) &= \frac{f(b)-f(a)}{b-a} \\
\end{align*}

\noindent \\
Include graph for clarity's sake.

% -- Bibliography (APA style)
\bibliography{references}

\end{document}