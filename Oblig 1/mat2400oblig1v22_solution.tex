\documentclass[12pt,letterpaper]{article}


\usepackage{amsmath}
\usepackage{amsfonts}
\usepackage{amsthm}
\usepackage{ulem}


\title{Mandatory Assignment 1 of 2}
\author{Håkon Berggren Olsen\\
  \small{MAT2400- Real Analisys}\\
  \small{University of Oslo}\\
  \small{hakonberggren@gmail.com}
}
\date{Spring 2022}

\begin{document}
\maketitle
\newpage 
 
\section*{Problem 1}
 
 Let $X$ be the set of all sequences $\{x_n\}_{n\in \mathbb{N}}$ of real numbers such that $lim_{n\to \infty} x_n = 0$.\\
 \noindent \\
 a) Use the definiton of convergence to show that if $\{x_n\}\in X$, then there is a $K\in \mathbb{N}$ such that $|x_K|=\text{sup}\{|x_n| : n\in \mathbb{N} \}$ (i.e. $x_K$ is an element of maximal absolute value).\\
 
 \noindent \\
 As all sequences $\{x_n\}_{n\in \mathbb{N}}$ in $X$ converges to $0$, $lim_{n\to \infty} x_n = 0$, we have that for every $\epsilon>0$ there exists a $N\in \mathbb{N}$ such that $|\{x_n\}-0| = |\{x_n\}|<\epsilon$. \\
 
 
 \begin{align*}
 	|x_K| = \text{sup} \{|x_n| \ : | n\in\mathbb{N}   \}
 \end{align*}
 
 
%All sequences in $X$ are real numbers and converges to $0$ and are therefore Cauchy sequences (Theorem 2.2.5 in \cite{lindstrom2017}, ''Completenes of $\mathbb{R}^m$ '' and Propostiion 2.2.6), and every Cauchy sequence in $\mathbb{R}$ is bounded. 
 
 
 \noindent
 \textbf{Definition of convergence:}\\
 ''A sequence $\{x_n\}$ of real numbers converges to $a\in\mathbb{R}$ if for every $\epsilon >0$ (no matter how small), there is an $N\in\mathbb{N}$ such that $|x_n-a|<"\epsilon$ for all $n\le N$. We write $lim_{n \to \infty} x_n = a$.''\\
 

 
 \noindent \\
 b) Define $d: X \times X \to [0, \infty)$ by
\begin{align*}
 	d(\{x_n\}, \{y_n\}) = \text{sup}\{|x_n-y_n| : n\in \mathbb{N}\}. 
\end{align*}
 Show that d is a metric on $X$.\\
 \noindent \\
For $(X,d)$ to be a metric, it ness to satisfy the properties of postivity, symmetry and the triangle inequality:
 \begin{itemize}
 	\item (Positivity) For all $x,y\in X$, we have $d(x,y)\ge 0$ with equalitiy if and only if $x=y$.
 	\item (Symmetry) For all $x,y \in X$ we have $d(x,y)=d(y,x)$. 
 	\item (Triangle Inequality) For all $x,y,z \in X$, we have
 		\begin{align*}
 			d(x,y) \le d(x,z)+d(z,y)
 		\end{align*}
 \end{itemize}

The first two properties are fairly obvious as $|x_n-y_n|\ge 0$ (postivity) and $|x_n-y_n| = |y_n-x_n|$ (symmetry). To prove the triangle inequality, first assume there exists $\{x_n\},\{y_n\},\{z_n\}\in X$. Looking at the argument for the supreme function we have the triangle inequality
\begin{align*}
	|x_n-y_n| &\le |x_n-z_n| + |y_n - z_n|
\end{align*}
further evaluating the supremum
\begin{align*}
	|x_n-z_n| + |y_n - z_n| &\le  \text{sup} \{|x_n-z_n| + |y_n - z_n|  : n\in\mathbb{N} \} \\
		&\le \text{sup} \{|x_n-z_n| : n\in\mathbb{N} \} +\text{sup} \{| z_n-y_n|  : n\in\mathbb{N} \} \\
		&= d(\{x_n\}, \{z_n\}) + d(\{z_n\}, \{y_n\})
\end{align*}
where the fact that $\text{sup} \{A+B\} = \text{sup} \{A\} + \text{sup} \{B\}$ is used. \\
This results in $d(\{x_n\}, \{z_n\}) + d(\{z_n\}, \{y_n\})$ being an upper bound \footnote{Not the least upper bound, hence the inequality.} for $|x_n-y_n|$ and we therefore get

 \begin{align*}
	 d(\{x_n\}, \{y_n\}) \le d(\{x_n\}, \{z_n\}) + d(\{z_n\}, \{y_n\})
\end{align*}

 \noindent \\
 c)  Let $Y$ be the set of all sequences $ \{y_n\}_{n\in\mathbb{N}}$ of real numbers such that $\sum_{n=1}^{\infty} |y_n| < \infty$. Show that $Y \subseteq X$. Find a sequence $\{x_n\}$ that belongs to $X$ but not to $Y$ (you can use everything you know from calculus).\\
 
 \noindent
As $X$ is the set of all sequences of real numbers that converges to $0$ (i.e. $lim_{n\to \infty} x_n = 0$) and $Y$ is the set of all sequences of real numbers such that $\sum_{n=1}^{\infty} |y_n| < \infty$, one functions that that belongs to $X$ but not to $Y$ is the harmonic series
\begin{align*}
	\sum_{n=1}^\infty \frac{1}{n}
\end{align*} 
This series converges to sero as $\lim_{n \to \infty} 1/n = 0$, but the sum of the function is not smaller than infinity.

 \noindent \\
d) Assume $\{x_n\} \in X \backslash Y$ and let $\epsilon > 0$. Show that the ball $B(\{x_n\}; \epsilon)$ contains elements from $Y$. Explain why this shows that  $Y$ is not closed.\\
 
 \noindent \\
 If the ball $B(\{x_n\}; \epsilon)$ contains elements, either has an interior- or boundary point in $Y$ then $Y$ is not closed. We can check this by firstly examining if there exists an exterior point of $Y$. \\
 \noindent \\
 If $B(\{x_n\}; \epsilon) \subset Y^c$ where $\{x_n\} \in X \backslash Y$.
 
\begin{itemize}
 	\item an interior point of $Y$ is there if $r>0$ such that $B(\{x_n\}; \epsilon) \subset Y$
	\item an exterior point of $Y$ is there if $r>0$ such that $B(\{x_n\}; \epsilon) \subset Y^c$
	\item a boundary point if of $A$ is there if all $r>0$ we have 
	\begin{align*}
		B(\{x_n\}, \epsilon) \cap A \neq \emptyset \quad and B(\{x_n\}, \epsilon) \cap A^c \neq \emptyset
	\end{align*}
\end{itemize}
 
 
 
 \textbf{\uline{Note:}} The ball is defined as such:\\
 \noindent \\ Let $a$ be a point in a metric space $(X,d)$, and assume that $r$ is a positive, real number. The (open) ball centered at $a$ with radius $r$ is set 
 \begin{align*}
 	\text{B}(a;r) = \{x\in X \ : \ d(x,a) <r   \}
 \end{align*}
 And the \emph{closed} ball centered at $a$ with radius $r$ is the set
 \begin{align*}
 	\overline{\text{B}}(a;r) = \{x\in X \ : \ d(x,a) \le r   \}
 \end{align*}
 
 
 
 \noindent \\
e) Assume $\{y_n\}\in Y$ and let $\epsilon >0$. Show that $B(\{y_n\};\epsilon)$ contains elements from $X \backslash Y$ Explain why this shows that $Y$ is not open. \\
 


 
 
\section*{Problem 2}

A metric space $(X,d)$ is called disconnected if there are two non-empty, open subsets $O_1, O_2$ such that $O_1 \cup O_2 = X$ and $O_1 \cap O_2 = \emptyset$. \\
%DONE
\noindent \\
a) Let $X = [0,1] \cup [2,3]$ have the usual metric $d(x,y) = |x-y|$. \\
\noindent \\
The metric space $(X,d)$ is composed of two closed sets. \\
By carefully splitting $X$ into $O_1= [0,\frac{1}{2}) \cup (\frac{3}{2},3]$ and $O_2 = (\frac{1}{2},1]\cup [2,\frac{3}{2})$, the ''inside'' and the ''outside'' of $X$, we now have two non-empty, open subsets which satisfy the properties of a disconnected metric space.
\begin{align*}
	O_1 \cup O_2  &= \left([0,\frac{1}{2}) \cup (\frac{3}{2},3]\right) \cup \left((\frac{1}{2},1]\cup [2,\frac{3}{2})\right)    = X \\
	O_1 \cap O_2  &= \left( [0,\frac{1}{2}) \cup (\frac{3}{2},3]\right) \cap \left( (\frac{1}{2},1]\cup [2,\frac{3}{2})\right)  = \emptyset
\end{align*}



\noindent \\
b) Show that $\mathbb{Q}$ with the usual metric $d(x,y)=|x-y|$ is disconnected. \\
(Hint: Consider $O_1=\{x\in\mathbb{Q} : x^2 > 2\}$ and $O_2=\{x\in\mathbb{Q} : x^2 > 2\}$.)\\

\noindent \\
Considering $O_1=\{x\in\mathbb{Q} : x^2 > 2\}$ and $O_2=\{x\in\mathbb{Q} : x^2 < 2\}$, which are open, non-empty subsets of $\mathbb{Q}$. The metric space is disconnected as the subset are disjointed

\begin{align*}
	O_1 \cup O_2 = \{x\in\mathbb{Q} : x^2 > 2\}\cap\{x\in\mathbb{Q} : x^2 < 2\} = \mathbb{Q} \\
	O_1 \cap O_2 = \{x\in\mathbb{Q} : x^2 > 2\}\cup\{x\in\mathbb{Q} : x^2 < 2\} = \emptyset \\
\end{align*}



\noindent \\
c) Assume that $(X,d)$ is a connected (i.e. not disconnected) metric space and that $f : X \to \mathbb{R}$ is a continous function such that there are two points $a,b \in X$ with $f(a)<0<f(b)$. Show that there is a point $c\in X$ such that $f(c)=0$.(This is an abstract version of the Intermediate Value Theorem.)





\end{document}